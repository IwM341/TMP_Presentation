Скорость захвата определяется интегралом
\begin{equation*}
	C = \int{d^3\vec{v}d^3\vec{v}_i d^3\vec{r}f_{0\chi}(\vec{v},\vec{r})f_{i}(\vec{v}_i,\vec{r})|\vec{v}-\vec{v}_i|d\sigma}
\end{equation*}
где $f_{i}$ --- функция расределения ядер типа $i$, $f_{0\chi}$ --- распределение частиц пришедших из гало (с плотностью $0.3 GeV/cm^3$).
Предполагается гауссово распределение скоростей со сдвигом из-за  движения тела.
Интеграл столкновений в координатах $E,L$ следующий

\begin{equation*}
	dN = \int{dEdL \cfrac{T_{in}}{T_{in}+T_{out}} d\tau f_{0\chi}(E,L)f_{i} d^3\vec{v}_i(\vec{v}_i,\vec{r})|\vec{v}-\vec{v}_i|d\sigma}
\end{equation*}
Сетка по $E, L$ --- прямоугольная, неравномерная (число разбиений по $L$ зависит от $E$ ) 