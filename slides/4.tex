В двухкомпонентной ТМ происходит нетривиальная термализация. Для ее расчета рассматривается движение в сферическом потенциале.
\begin{equation*}	
	H = \frac{v_{r}^{2}}{2} + \left( {\phi(r) + \frac{L^{2}}{2r^{2}}} \right) = \frac{v_{r}^{2}}{2} + U_{eff}(L,r)
\end{equation*}
В уравнении Больцмана
\begin{equation*}
	\frac{df}{dt} = C\left( {\vec{r},\vec{v}} \right) + St \left[ f( \vec{r},\vec{v}')\right] \left( {\vec{r},\vec{v}} \right)
\end{equation*}
идет переход к параметрам $E,L$ - энергия и момент импульса.

